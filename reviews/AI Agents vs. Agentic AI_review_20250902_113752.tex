
\documentclass[11pt]{article}
\usepackage[utf8]{inputenc}
\usepackage{geometry}
\geometry{margin=1in}
\usepackage{amsmath}
\usepackage{cite}

\title{Peer Review Report}
\author{AI Reviewer (Demo Mode)}
\date{\today}

\begin{document}
\maketitle

\section{Executive Summary}
This paper contains approximately 24293 words and presents research findings in a structured format. The work demonstrates theoretical concepts with visual aids including figures and tabulated data.

\section{Strengths}
\begin{itemize}
\item Clear problem formulation and research objectives
\item Well-structured theoretical framework
\item Good integration of visual elements (figures/tables)
\item Adequate scope for the research domain
\end{itemize}

\section{Areas for Improvement}
\begin{itemize}
\item Consider expanding the literature review section
\item Provide more detailed methodology description
\item Include additional experimental validation
\item Improve discussion of limitations
\end{itemize}

\section{Technical Comments}
\subsection{Methodology}
The paper would benefit from more detailed explanation of the experimental setup and data collection procedures.

\subsection{Results}
The presented results are reasonable but require more comprehensive analysis.

\section{Minor Issues}
\begin{itemize}
\item Check formatting consistency throughout the document
\item Verify all references are properly cited
\item Consider improving figure/table captions for clarity
\end{itemize}

\section{Overall Assessment}
This paper presents valuable insights that contributes to the field. With the suggested improvements, particularly in methodology description and validation, this work has potential for publication.

\textbf{Recommendation:} Minor to moderate revisions required.

\end{document}
